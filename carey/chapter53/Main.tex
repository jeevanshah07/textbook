\addcontentsline{toc}{subsection}{Improper Integrals}
\subsection*{Improper Integrals}
\textbf{Recall} the following two conditions from theorems earlier this year:
\begin{enumerate}

    \item The definition of the definite integral requires that the interval $[a,\,b]$ is finite.
    \item The Fundamental Theorem of Calculus requires that the function is continuous on $[a,\,b]$.
\end{enumerate}
Integrals that do not possess these properties are called \textbf{improper integrals}, and we can still compute them with the addition of limits!

\begin{tcolorbox}[title= DEFINITION OF IMPROPER INTEGRALS WITH INFINITE BOUNDS,colframe=black,sharp corners,colback=white,colbacktitle=white,coltitle=black]

    \begin{enumerate}
        \item If $f$ is continuous on the interval $[a,\,\infty)$, then 
        \[\int_a^\infty f(x)\,dx=\lim_{b\to\infty}\int_a^b f(x)\,dx.\]
        \item If $f$ is continuous on the interval $(-\infty,\,b]$, then 
        \[\int_{-\infty}^b f(x)\,dx=\lim_{a\to-\infty}\int_a^b f(x)\,dx.\]
        \item If $f$ is continuous on the interval $(-\infty,\,\infty)$, then 
        \[\int_{-\infty}^\infty f(x)\,dx=\int_{-\infty}^c f(x)\,dx + \int_c^{\infty} f(x),\,dx\]
        where $c$ is any real number.
    \end{enumerate}
    
    If the limit exists, then it is said that improper integral \textbf{converges}, otherwise, it \textbf{diverges}. In the third case, if either of the integrals on the right diverge, then so does the one on the left.

\end{tcolorbox}

\textbf{Examples:}
\begin{questions}
    \question Evaluate $\displaystyle\int_1^\infty \frac{1}{x}\,dx$.
    \vspace{\stretch{1}}
    
    \newpage
    
    \begin{minipage}{.45\linewidth}
        \question $\displaystyle\int_1^\infty e^{-x}\,dx$
    \end{minipage}
    \hfill
    \begin{minipage}{.45\linewidth}
        \question $\displaystyle\int_1^\infty (1-x)e^{-x} \,dx$
    \end{minipage}

    \vspace{\stretch{1}}
    
    
    \question $\displaystyle\int_{-\infty}^\infty\frac{e^x}{1+e^{2x}}\,dx$

    \vspace{\stretch{1}}

\end{questions}

\newpage


\begin{tcolorbox}[title= DEFINITION OF IMPROPER INTEGRALS WITH INFINITE DISCONTINUITIES,colframe=black,sharp corners,colback=white,colbacktitle=white,coltitle=black]

    \begin{enumerate}
        \item If $f$ is continuous on the interval $[a,b)$ and has an infinite discontinuity at $b$, then 
        \[\int_a^b f(x)\,dx=\lim_{c\to b^-}\int_a^c f(x)\,dx.\]
        \item If $f$ is continuous on the interval $(a,b]$ and has an infinite discontinuity at $a$, then 
        \[\int_a^b f(x)\,dx=\lim_{c\to a^+}\int_c^b f(x)\,dx.\]
        \item If $f$ is continuous on the interval $[a,b]$ except for some $c$ in $(a,b)$ at which $f$ has an infinite discontinuity, then 
        \[\int_a^b f(x)\,dx=\int_{a}^c f(x)\,dx + \int_c^{b} f(x),\,dx.\]
    \end{enumerate}
    
    If the limit exists, then it is said that improper integral \textbf{converges}, otherwise, it \textbf{diverges}. In the third case, if either of the integrals on the right diverge, then so does the one on the left.

\end{tcolorbox}


\textbf{Examples:}
\begin{questions}
    \begin{minipage}{.45\linewidth}
        \question Evaluate $\displaystyle\int_0^1\frac{dx}{\sqrt[3]{x}}$.    
    \end{minipage}
    \hfill
    \begin{minipage}{.45\linewidth}
        \question Evaluate $\displaystyle\int_0^2\frac{dx}{x^3}$.
    \end{minipage}
    
    \vspace{\stretch{1}}
    
    \newpage
    
    \question Evaluate $\displaystyle\int_{-1}^2\frac{dx}{x^3}$.
    
    \vspace{\stretch{1}}
    
    
    
    \question Evaluate $\displaystyle\int_0^\infty\frac{dx}{\sqrt{x}(x+1)}.$
    
    \vspace{\stretch{1}}
\end{questions}


\begin{tcolorbox}[title= A SPECIAL IMPROPER INTEGRAL,colframe=black,sharp corners,colback=white,colbacktitle=white,coltitle=black]

    \[\int_1^\infty\frac{dx}{x^p}=
    \begin{cases}
        \frac{1}{p-1} & \text{if }p>1\\
        \text{diverges} & \text{if }p\le1
    \end{cases}\]

\end{tcolorbox}


\newpage
