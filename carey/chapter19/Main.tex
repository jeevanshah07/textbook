\addcontentsline{toc}{subsection}{Logarithmic Differentiation}
\subsection*{Logarithmic Differentiation}
For the two previous problems, we could use properties of logs to make life much easier. Wouldn't it be great to apply those same rules to other hard problems such as $\displaystyle y=\sqrt[3]{\frac{x+1}{x-1}}$? We can, just \textbf{take the }$\ln$\textbf{ of both sides!}
\begin{questions}
    \question $\displaystyle y=x^x,\, x>0$
    
    \vspace{\stretch{1}}
    
    \question $y=x^{\tan x}$
    
    \vspace{\stretch{1}}
    
    \question $\displaystyle y^5=\sqrt{\frac{(x+1)^5}{(x+2)^{10}}}$
    
    \vspace{\stretch{1}}
    
    \question $\displaystyle y^2=\frac{x\sqrt{x^2+1}}{(x+1)^{2/3}}$
    
    \vspace{\stretch{1}}
\end{questions}




\newpage
