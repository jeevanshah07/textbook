\addcontentsline{toc}{subsection}{Conditional and Absolute Convergence}
\subsection*{Absolute and Conditional Convergence}
Occasionally, we run into series that have both positive and negative terms but is \textbf{not} an alternating series. For example:



\[\displaystyle\sum_{n=1}^{\infty}\frac{\sin n}{n^2}=\frac{\sin1}{1}+\frac{\sin2}{4}+\frac{\sin3}{9}+\cdots\]
In order to obtain any information about the convergence of this series, let's instead consider
\[\displaystyle\sum_{n=1}^{\infty}\left|\frac{\sin n}{n^2}\right|.\]
\vspace{\stretch{1}}

\begin{tcolorbox}[title= ABSOLUTE CONVERGENCE,colframe=black,sharp corners,colback=white,colbacktitle=white,coltitle=black]

    If the series $\displaystyle\sum\left|a_n\right|$ converges, then the series $\displaystyle\sum a_n$ also converges.

\end{tcolorbox}
\vspace{.1in}
Note that the opposite is \textbf{not} true. For instance, the \textbf{alternating harmonic series} converges by the alternating series test. Yet the harmonic series diverges. This is called \textbf{conditional convergence.}
\[\sum_{n=1}^\infty \frac{(-1)^{n+1}}{n}=\frac{1}{1}-\frac{1}{2}+\frac{1}{3}-\frac{1}{4}+\cdots\]


\newpage

\begin{tcolorbox}[title= DEFINITIONS OF ABSOLUTE AND CONDITIONAL CONVERGENCE,colframe=black,sharp corners,colback=white,colbacktitle=white,coltitle=black]

    \begin{enumerate}
        \item $\displaystyle\sum a_n$ is \textbf{absolutely convergent} if $\displaystyle\sum\left|a_n\right|$ converges.
        \item $\displaystyle\sum a_n$ is \textbf{conditionally convergent} if $\displaystyle\sum a_n$ converges but $\displaystyle\sum\left|a_n\right|$ diverges.
    \end{enumerate}

\end{tcolorbox}
\vspace{.1in}
\noindent\textbf{Practice:}\\
Determine whether each of the series is convergent or divergent. Classify any convergent series as absolutely or conditionally convergent.
\begin{questions}
    
    \begin{minipage}{0.45\linewidth}
    \question $\displaystyle\sum_{n=0}^\infty \frac{(-1)^n n!}{2^n}$
    \end{minipage}
    \hfill
    \begin{minipage}{0.45\linewidth}
    \question $\displaystyle\sum_{n=1}^\infty \frac{(-1)^n}{\sqrt{n}}$
    \end{minipage}
    \vspace{\stretch{1}}
    
    \begin{minipage}{0.45\linewidth}
    \question $\displaystyle\sum_{n=1}^\infty \frac{(-1)^{n(n+1)/2} n!}{3^n}$
    \end{minipage}
    \hfill
    \begin{minipage}{0.45\linewidth}
    \question $\displaystyle\sum_{n=1}^\infty \frac{(-1)^n}{\ln(n+1)}$
    \end{minipage}
    \vspace{\stretch{1}}
    
    
\end{questions}


\newpage
