\addcontentsline{toc}{subsection}{The Chain Rule}
\subsection*{The Chain Rule}
In your head, consider the derivative of the function $y=(3x+4)^2$. To do it, we have to "take the derivative of the inside". More formally, this is another differentiation rule that we call the \textbf{Chain Rule}.\\

\begin{tcolorbox}[title= THE CHAIN RULE,colframe=black,sharp corners,colback=white,colbacktitle=white,coltitle=black,boxrule=1pt]

    If $y=f(u)$ is a differentiable function of $u$ and $u=g(x)$ is a differentiable function of $x$, then $y=f(g(x))$ is a differentiable function of $x$ and 
    \[\frac{dy}{dx}=\frac{dy}{du}\cdot\frac{du}{dx}\]
    or, equivalently
    \[\frac{d}{dx}\left(f(g(x))\right)=f'(g(x))\cdot g'(x).\]
    
\end{tcolorbox}
\vspace{.15cm}
\noindent\textbf{Examples:}
\begin{questions}
    \question Suppose that functions $f$ and $g$ and their derivatives have the following values at $x=2$ and $x=3$.
    \begin{longtable}[ht]{|C{1.5cm}|C{1.5cm}|C{1.5cm}|C{1.5cm}|C{1.5cm}|}
        \hline
        $x$\Tstrut\Bstrut & $f(x)$\Tstrut\Bstrut & $g(x)$\Tstrut\Bstrut & $f'(x)$\Tstrut\Bstrut & $g'(x)$\Tstrut\Bstrut\\\hline
        $2$\Tstrut\Bstrut & $8$\Tstrut\Bstrut & $2$\Tstrut\Bstrut & $1/3$\Tstrut\Bstrut & $-3$\Tstrut\Bstrut\\\hline
        $3$\Tstrut\Bstrut & $3$\Tstrut\Bstrut & $-4$\Tstrut\Bstrut & $2\pi$\Tstrut\Bstrut & $5$\Tstrut\Bstrut\\\hline
    \end{longtable}
    Evaluate  the derivatives with respect to $x$ of the following combinations at the given value of $x$
    \begin{parts}
        \begin{minipage}{.45\linewidth}
            \part $\displaystyle 2f(x),\, x=2 $
        \end{minipage}
        \hfill
        \begin{minipage}{.45\linewidth}
            \part $\displaystyle f(x)\cdot g(x),\, x=3$
        \end{minipage}
        
        \vspace{\stretch{1}}
        
        \begin{minipage}{.45\linewidth}
            \part $\displaystyle \frac{f(x)}{g(x)},\, x=2$
        \end{minipage}
        \hfill
        \begin{minipage}{.45\linewidth}
            \part $\displaystyle f(g(x)),\,x=2$
        \end{minipage}
        
        \vspace{\stretch{1}}
        
        \begin{minipage}{.45\linewidth}
            \part $\displaystyle \frac{1}{g^2(x)},\,x=3$
        \end{minipage}
        \hfill
        \begin{minipage}{.45\linewidth}
            \part $\displaystyle \sqrt{f^2(x)+g^2(x)},\,x=2$
        \end{minipage}
        
        \vspace{\stretch{1}}
    \end{parts}
    
    \newpage
    
    \question Find $\displaystyle\frac{dy}{dx}$ for the two similar functions.
    \begin{parts}
        \begin{minipage}{.45\linewidth}
            \part $\displaystyle \frac{d}{dx}\left(\sin^2 x\right)=$
        \end{minipage}
        \hfill
        \begin{minipage}{.45\linewidth}
            \part $\displaystyle \frac{d}{dx}\left(\sin x^2\right)=$
        \end{minipage}
    \end{parts}
    
    \vspace{\stretch{1}}
    
    \question $\displaystyle f(u)=\left(\frac{u-1}{u+1}\right)^2,\, g(x)=\frac{1}{x^2}-1$ and find $(f\circ g)'(-1)$.
    
    \vspace{\stretch{1}}
    
    \question Let $y=x^2+7x-5.$ Evaluate $\displaystyle\frac{dy}{dt}$ when $x=1$ and $\displaystyle\frac{dx}{dt}=\frac{1}{3}$.
    
    \vspace{\stretch{1}}
    

    
\end{questions}


\newpage
