\addcontentsline{toc}{subsection}{Alternating Series}
\subsection*{Alternating Series}
The simplest of alternating series can be a geometric series. The following example is easy to determine its convergence since we know that the geometric series test can be applied:
\[\sum_{n=0}^{\infty}\left(-\frac{1}{2}\right)^n=1-\frac{1}{2}+\frac{1}{4}-\frac{1}{8}+\frac{1}{16}-\cdots\]

\vspace{.07in}

\begin{tcolorbox}[title= THE ALTERNATING SERIES TEST,colframe=black,sharp corners,colback=white,colbacktitle=white,coltitle=black]

    Let $a_n>0$. The alternating series
    \[\sum_{n=1}^\infty(-1)^n\, a_n\hspace{.15in}\text{and}\hspace{.15in}\sum_{n=1}^\infty(-1)^{n+1}\,a_n\]
    converge if the following conditions are met.\\
    \begin{enumerate}
        \item $\displaystyle\lim_{n\to\infty}a_n=0$
        \item $\displaystyle a_{n+1}\le a_n$, for all $n$
    \end{enumerate}

\end{tcolorbox}
\vspace{.1in}
\noindent\textbf{ONE IMPORTANT FACT}:
\begin{questions}
    \question The second condition in the Alternating series test can be modified to require only that $0<a_{n+1}\le a_n$ for all $n$ greater than some integer $N$.
\end{questions}
\vspace{.1in}

\noindent\textbf{Examples:}\\
Determine the convergence or divergence of each series.
\begin{parts}
    \begin{minipage}{0.45\linewidth}
        \part $\displaystyle\sum_{n=1}^\infty (-1)^{n+1}\frac{1}{2+3^n}$
    \end{minipage}
    \hfill
    \begin{minipage}{0.45\linewidth}
        \part $\displaystyle\sum_{n=1}^\infty \frac{n}{(-2)^{n-1}}$
    \end{minipage}
\end{parts}

\newpage



\noindent\textbf{Example: When the AST Fails}
\begin{questions}
    \question $\displaystyle\sum_{n=1}^\infty \frac{(-1)^{n+1}(n+1)}{n}=\frac{2}{1}-\frac{3}{2}+\frac{4}{3}-\frac{5}{4}+\frac{6}{5}-\cdots$
    \vspace{\stretch{1}}
    
    \question $\displaystyle\frac{2}{1}-\frac{1}{1}+\frac{2}{2}-\frac{1}{2}+\frac{2}{3}-\frac{1}{3}+\frac{2}{4}-\frac{1}{4}+\cdots$
    \vspace{\stretch{1}}
\end{questions}



\noindent\textbf{\Large Alternating Series Remainder:}
\begin{tcolorbox}[title= THE ALTERNATING SERIES REMAINDER THEOREM,colframe=black,sharp corners,colback=white,colbacktitle=white,coltitle=black]

    If a convergent alternating series satisfies the condition $\displaystyle a_{n+1}\le a_n$, then the absolute value of the reminder $R_N$ involved in approximating the sum $S$ by $S_N$ is less than (or equal to) the first neglected term. That is,
    \[\left|S-S_N\right|=\left|R_N\right|\le a_{N+1}\]

\end{tcolorbox}
\vspace{.1in}
The reason why this is the case is hard to conceptualize, so perhaps a proof might help.\\
\\
\noindent \textbf{Proof:} The series obtained by deleting the first $N$ terms of the given series satisfies the conditions of the conditions of the Alternating series test and has a sum of $R_N$.
\begin{align*}
    R_N = S-S_N &= \displaystyle\sum_{n=1}^{\infty}(-1)^{n+1}a_n-\sum_{n=1}^{N}(-1)^{n+1}a_n\\
     &= \displaystyle(-1)^{N}a_{N+1} + (-1)^{N+1}a_{N+2} + (-1)^{N+2}a_{N+3} + \cdots\\
     &= \displaystyle(-1)^{N}\left(a_{N+1}-a_{N+2}+a_{N+3}-\cdots\right)\\
    |R_N| &= a_{N+1}-a_{N+2}+a_{N+3}-a_{N+4}+a_{N+5}-\cdots\\
    &= a_{N+1}-\left(a_{N+2}-a_{N+3}\right)-\left(a_{N+4}-a_{N+5}\right)-\cdots\le a_{N+1}
\end{align*}
Consequently, $\displaystyle\left|S-S_N\right|=|R_N|\le a_{N+1}$, which establishes the theorem.



\newpage

\noindent\textbf{Examples:}\\
\begin{questions}
    \question Consider the following series:
    \[\displaystyle\sum_{n=1}^{\infty}(-1)^{n+1}\left(\frac{1}{n!}\right)=\frac{1}{1!}-\frac{1}{2!}+\frac{1}{3!}-\frac{1}{4!}+\frac{1}{5!}-\frac{1}{6!}+\cdots\]
    \begin{parts}
        \part Approximate the sum of the series using the first six terms.
        
        
        
        \part What is the error associated with this approximation?
        
        
        
        \part What is the interval for which the true sum of the series, $S$, must lie within? (Use the previous two answers)
        
        \vspace{\stretch{1}}
    \end{parts}
    
    \question Consider the series $\displaystyle\sum_{n=0}^\infty \frac{(-1)^n \cdot 3^{n+1}}{2^{2n}}$.
    \begin{parts}
        \part Find the maximum error associated with the approximation of $S_4$.
        \part How many terms would it take for the error to be less than $\frac{1}{200}$?
    \end{parts}
    
    
    \vspace{\stretch{1}}
    
    
    
    
\end{questions}





\newpage


