\addcontentsline{toc}{subsection}{Rules for Differentiation A}
\subsection*{Rules for Differentiation A}

Let $y=f(x)$ be a function of $x$. If the limit $\displaystyle\lim_{\Delta x\to0}\frac{f(x+\Delta x)-f(x)}{\Delta x}$ exists and is finite, we call this limit the derivative of $f$ at $x$ and say that $f$ is \textit{differentiable} at $x$.

\noindent\textbf{Rules for Differentiation:}
\begin{questions}
    \question \textbf{Constant Rule:}
    \begin{parts}
        \part $\displaystyle y=5$
    \end{parts}
    
    \question \textbf{Power Rule:}
    \begin{parts}
        \part $\displaystyle y=x^2$
        \part $\displaystyle f(x)=x^3$
        \part $\displaystyle \frac{d}{dx}\left(x^4\right)=$
        \part $\displaystyle y=x^{13}$
        \part $\displaystyle \frac{d}{dx}\left(x^{79}\right)=$
    \end{parts}
    
    \question \textbf{Constant Multiple Rule:}
    \begin{parts}
        \part $\displaystyle \frac{d}{dx}\left(7x^5\right)=$
        \part $\displaystyle y=-3x^4$
        \part $\displaystyle \frac{d}{dx}\left(\frac{2}{3}x^3\right)=$
    \end{parts}
    
    \question \textbf{Sum and Difference Rule:}
    \begin{parts}
        \part $\displaystyle \frac{d}{dx}\left(x^3+7x^2-5x+4\right)=$
        \part $\displaystyle y= \frac{1}{4}x^4-\frac{x^3}{3}+4$
    \end{parts}
    
    \question \textbf{Higher Order Derivatives:}
    \begin{parts}
        \part $\displaystyle y=x^3-5x^2+2$\vspace{\stretch{1}}
        
        \part $\displaystyle y=\frac{x+1}{x}$
        \vspace{\stretch{1}}
    \end{parts}
\end{questions}


\newpage

For a derivative to exist, the left-hand and right-hand derivatives must be the same in order for there to be derivative at that point \textit{and} for derivatives, it must be continuous at that point.
\begin{questions}
    \question Determine whether the curve has a tangent at the indicated point. If it does, give the slope at that point. If not, explain why.
    \begin{parts}
        \part $\displaystyle f(x)=\begin{cases}
        -x      & x<0\\
        x^2-x   & x\ge0
        \end{cases}$ at $x=0$
        \vspace{\stretch{1}}
        \part $\displaystyle f(x)=\begin{cases}
        \sin x      & 0\le x<3\pi/4\\
        \cos x   & 3\pi/4\le x\le 2\pi
        \end{cases}$ at $x=3\pi/4$
        \vspace{\stretch{1}}
    \end{parts}
    \question True or False: The graph of $f(x)=|x|$ has a tangent line at x=0. Justify your answer.
    \vspace{\stretch{.67}}
\end{questions}



\newpage
