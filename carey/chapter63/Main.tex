\addcontentsline{toc}{subsection}{Ratio and Root Tests}
\subsection*{Ratio and Root Tests}

\begin{tcolorbox}[title= THE RATIO TEST,sharp corners,colback=white,colbacktitle=white,coltitle=black]

    Let     $\displaystyle\sum a_n$ be a series with nonzero terms.
    \begin{enumerate}
        \item $\displaystyle\sum a_n$ converges absolutely if $\displaystyle\lim_{n\to\infty}\left|\frac{a_{n+1}}{a_n}\right|<1$.
        \item $\displaystyle\sum a_n$ diverges if $\displaystyle\lim_{n\to\infty}\left|\frac{a_{n+1}}{a_n}\right|>1$ or $\displaystyle\lim_{n\to\infty}\left|\frac{a_{n+1}}{a_n}\right|=\infty$.
        \item The Ratio Test is inconclusive if $\displaystyle\lim_{n\to\infty}\left|\frac{a_{n+1}}{a_n}\right|=1$.
    \end{enumerate}

\end{tcolorbox}
\vspace{.1in}
\noindent\textbf{ONE IMPORTANT FACT}:
\begin{questions}
    \question This is not the one stop shop for tests of convergence, but \textit{usually} is for series involving factorials or exponentials.
\end{questions}
\vspace{.1in}

\noindent\textbf{Example:}\\
Determine the convergence or divergence of each series.
\begin{parts}
    \begin{minipage}{0.45\linewidth}
        \part $\displaystyle\sum_{n=0}^\infty \frac{2^n}{n!}$
    \end{minipage}
    \hfill
    \begin{minipage}{0.45\linewidth}
        \part $\displaystyle\sum_{n=1}^\infty \frac{n^n}{n!}$
    \end{minipage}
    
    \vspace{\stretch{1}}
    
    \part $\displaystyle\sum_{n=1}^\infty (-1)^n \frac{\sqrt{n}}{n+1}$
    
    \vspace{\stretch{1}}
    
\end{parts}

\newpage

\begin{tcolorbox}[title= THE ROOT TEST,colframe=black,sharp corners,colback=white,colbacktitle=white,coltitle=black]

    Let
    $\displaystyle\sum a_n$ be a series.
    \begin{enumerate}
        \item $\displaystyle\sum a_n$ converges absolutely if $\displaystyle\lim_{n\to\infty}\sqrt[n]{\left|a_n\right|}<1$.
        \item $\displaystyle\sum a_n$ diverges if $\displaystyle\lim_{n\to\infty}\sqrt[n]{\left|a_n\right|}>1$ or $\displaystyle\lim_{n\to\infty}\sqrt[n]{\left|a_n\right|}=\infty$.
        \item The Root Test is inconclusive if $\displaystyle\lim_{n\to\infty}\sqrt[n]{\left|a_n\right|}=1$.
    \end{enumerate}

\end{tcolorbox}
\vspace{.1in}

\noindent\textbf{ONE IMPORTANT FACT}:
\begin{questions}
    \question The Root Test is always inconclusive for a $p$-series. So don't even bother.
\end{questions}
\vspace{.1in}
\noindent\textbf{Example:}\\
Test the series for convergence or divergence. Afterwards, \textit{try} the Ratio Test to see why it is not ideal.
$\displaystyle\sum_{n=1}^\infty \frac{e^{2n}}{n^n}$


\newpage

\noindent\textbf{\Large Strategies for Testing Series}\\
\\
You have learned 10 strategies for testing a series for convergence or divergence. 

\textbf{Guidelines for Testing a Series:}
\begin{enumerate}
    \item Does the $n$th term approach 0? If not, the series diverges.
    \item Is the series one of the special types (geo, $p$-series, telescoping, alternating)?
    \item Can the integral test, ratio test, or root test be applied?
    \item Can the series be compared favorably to one of the special types?
\end{enumerate}

\noindent\textbf{Example: Applying Strategies}
\begin{parts}
    \begin{minipage}{0.45\linewidth}
        \part $\displaystyle\sum_{n=1}^{\infty}\frac{n+1}{3n+1}$
    \end{minipage}
    \hfill
    \begin{minipage}{0.45\linewidth}
        \part $\displaystyle\sum_{n=1}^{\infty}\left(\frac{\pi}{6}\right)^n$
    \end{minipage}
    
    \vspace{\stretch{1}}
    
    \begin{minipage}{0.45\linewidth}
        \part $\displaystyle\sum_{n=1}^{\infty}ne^{-n^2}$
    \end{minipage}
    \hfill
    \begin{minipage}{0.45\linewidth}
        \part $\displaystyle\sum_{n=1}^{\infty}\frac{1}{3n+1}$
    \end{minipage}
    
    \vspace{\stretch{1}}
    
    \newpage
    
    \begin{minipage}{0.45\linewidth}
        \part $\displaystyle\sum_{n=1}^{\infty}(-1)^n\frac{3}{4n+1}$
    \end{minipage}
    \hfill
    \begin{minipage}{0.45\linewidth}
        \part $\displaystyle\sum_{n=1}^{\infty}\frac{n!}{10^n}$
    \end{minipage}
    \vspace{\stretch{1}}
    \part $\displaystyle\sum_{n=1}^{\infty}\left(\frac{n+1}{2n+1}\right)^n$
    \vspace{\stretch{1}}


\end{parts}


\newpage
