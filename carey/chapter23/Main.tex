\addcontentsline{toc}{subsection}{Mean Value Theorem}
\subsection*{Mean Value Theorem \textit{\tiny (for derivatives)}}

\begin{tcolorbox}[title= THE MEAN VALUE THEOREM,colframe=black,sharp corners,colback=white,colbacktitle=white,coltitle=black,boxrule=1pt]

    If $f$ is continuous on the closed interval $[a,\,b]$ and differentiable in the open interval $(a,\,b)$, then there exists a number $c$ in $(a,\,b)$ such that
    \[f'(c)=\frac{f(b)-f(a)}{b-a}.\]
    
\end{tcolorbox}
\vspace{2cm}

\textit{Think about it graphically:}
\vspace{2cm}

Another example: if a car accelerating from 0 takes 8 seconds to go 352 ft, its average velocity for the 8 second interval is 44 ft/sec (30 mph). At some point during the acceleration, the speedometer \textit{must} read exactly 30 mph!\\
\\
\textbf{Examples:} For each of the following functions on the indicated intervals, determine whether or not the MVT applies. If it does, find the value(s) where the mean value is attained.
\begin{questions}
    \question $\displaystyle f(x)=x^3+1$ on $[1,\,2]$
    \vspace{\stretch{1}}
    
    \question $\displaystyle f(x)=\sqrt{x^2}+1$ on $[-1,\,1]$
    \vspace{\stretch{1}}
    
    \question $\displaystyle f(x)=\frac{3}{x+2}$ on $[-3,\,1]$
    \vspace{\stretch{1}}
    
    \question $\displaystyle f(x)=\begin{cases}
    x^3+3 & x<1\\ x^2+1 & x\ge1
    \end{cases}$ on $[-2,\,2]$
    \vspace{\stretch{1}}
\end{questions}



\newpage
