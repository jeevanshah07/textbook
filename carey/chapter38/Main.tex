\addcontentsline{toc}{subsection}{The Fundamental Theorem of Calculus}
\subsection*{The Fundamental Theorem of calculus}
The \textbf{indefinite integral} is a family of functions. $F(x)$ is the antiderivative of $f(x)$.
\[\int f(x)\,dx=F(x)+C\]
The \textbf{definite integral} is a number. The number represents the net area between the graph of a function and the $x$-axis. \textit{Note: areas below the x-axis are negative.}
\[\int_a^b f(x)\,dx=F(b)-F(a)\]
The \textbf{definite integral as a function of \textit{x}} is a function. It is called the \textbf{accumulation function}. You can think of $F$ as accumulating the area under the graph of $f$ as the values of $t$ increase from $a$ to $x$. \textit{Imagine a paint roller!}
\[F(x)=F(a)+\int_a^x f(t)\,dt\]

\textbf{Examples:}
\begin{questions}
    \question Write the antiderivative F for the given function, $f$.
    \begin{parts}
        \begin{minipage}{0.45\linewidth}
            \part $\displaystyle f(x)=-\sin x$
        \end{minipage}
        \hfill
        \begin{minipage}{0.45\linewidth}
            \part $\displaystyle f(x)=-\sin\left(x^2\right)$ so that $F(0)=3$.
        \end{minipage}
    \end{parts}
    
    \vspace{\stretch{1}}
    
    \question Construct a function of the form $y=\displaystyle C+ \int_a^x f(t)\,dt$ with derivative $\displaystyle\frac{dy}{dx}=\tan x$ that satisfies the condition $F(3)=5.$
    \vspace{\stretch{1}}
    
\end{questions}

\newpage

\begin{tcolorbox}[title= DERIVATIVE OF THE INTEGRAL,colframe=black,sharp corners,colback=white,colbacktitle=white,coltitle=black,boxrule=1pt]

     If $f$ is continuous on $[a,\,b]$, then the function $\displaystyle F(x)=\int_a^x f(t)\,dt$ has a derivative at every point $x$ in $[a,\,b]$, and
     \[\frac{dF}{dx}=\frac{d}{dx}\int_a^x f(t)\,dt=f(x)\].
    
\end{tcolorbox}
\begin{center}
    IMPORTANT NOTE: Chain Rules does apply to this process.
\end{center}

\textbf{Examples:}
\begin{questions}
    \question $\displaystyle\frac{d}{dx}\int_1^x (2t+1)\,dt$
    \vspace{\stretch{.5}}
    
    \question $\displaystyle\frac{d}{dx}\int_{-2}^x \sqrt{1+e^{5t}}\,dt$
    \vspace{\stretch{.5}}
    
    \question $\displaystyle\frac{d}{dx}\int_x^{5}3t\sin t\,dt$
    \vspace{\stretch{.5}}
    
    \question Find $\displaystyle\frac{dy}{dx}$ for $y=\displaystyle\int_1^{x^2}\cos t\,dt$
    \vspace{\stretch{.5}}
    
    \question Find $\displaystyle\frac{dy}{dx}$ for $y=\displaystyle\int_{2x}^{x^2}\frac{1}{2+e^t}\,dt$
    \vspace{\stretch{.5}}
    
    
    \textbf{Rewrite the above definition using a $u$-substitution notation:}
    
    \vspace{1in}
    
    
    \newpage
    
    \question Let $g(x)=\displaystyle\int_1^x\left(5-8\sqrt{\ln t}\right)\,dt$ for $x>1$. Let $h(x)=\displaystyle\int_1^{x^2}\left(5-8\sqrt{\ln t}\right)\,dt$ for $x>1$.
    
    \begin{parts}
        \part Write an equation of the line tangent to $g$ at $x=3$.
        \vspace{\stretch{1}}
        
        \part What is $h'(x)$?
        \vspace{\stretch{1}}
        
        \part On which open interval(s) is $g$ decreasing? Justify your answer. \texttt{(Calculator allowed)}
        \vspace{\stretch{1.3}}
        
        \part Find all values of $x$ for which $h$ has relative extrema. Label them as maximum of minimum and justify your answer.
        \vspace{\stretch{1.3}}
    \end{parts}

\end{questions}




\newpage
