\addcontentsline{toc}{subsection}{Power Series}
\subsection*{Power Series}
In the previous section, we have been approximating some of our elementary functions using Taylor and Maclaurin Polynomials. We examined polynomials of first, second, third (and so on) degree and saw how closely they matched the original curve. The goal is going to be to move away from \textit{approximations} and instead represent a function \textit{exactly}.\\
\\
For example the function $f(x)=e^x$ can be represented exactly by an infinite series called a \textbf{power series}. The representation is
\[e^x=1+x+\frac{x^2}{2!}+\frac{x^3}{3!}+\cdots+\frac{x^n}{n!}+\cdots.\]
For each real number $x$, it can be shown that the infinite series on the right converges to the number $e^x$.\\


\begin{tcolorbox}[title= DEFINITIONS OF POWER SERIES,colframe=black,sharp corners,colback=white,colbacktitle=white,coltitle=black]

    If $x$ is a variable, then an infinite series of the form
    \[\sum_{n=0}^{\infty}a_n x^n=a_0 + a_1 x+a_2 x^2+a_3 x^3+\cdots+a_n x^n+\cdots\]
    is called a \textbf{power series}. More generally, an infinite series of the form
    \[\sum_{n=0}^{\infty}a_n (x-c)^n=a_0 + a_1 (x-c)+a_2 (x-c)^2+a_3 (x-c)^3+\cdots+a_n (x-c)^n+\cdots\]
    is called a \textbf{power series centered at \textit{c}}, where $c$ is a constant.

\end{tcolorbox}
\vspace{.1in}

