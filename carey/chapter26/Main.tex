\addcontentsline{toc}{subsection}{L'Hopital's Rule}
\subsection*{L'Hopital's Rule}
Remember when we computed $\displaystyle\lim_{x\to0}\frac{\sin x}{x}$? We got $\frac{0}{0}$, which is called an \textbf{indeterminate form}. While we could use Squeeze Theorem to show that this limit becomes 1, it would be easier to show using \textbf{L'Hopital's Rule}.

\begin{tcolorbox}[title= L'HOPITAL'S RULE,colframe=black,sharp corners,colback=white,colbacktitle=white,coltitle=black,boxrule=1pt]

    Suppose that $f(a)=g(a)=0$, that $f'(a)$ and $g'(a)$ exist, and that $g'(x)\ne0$. Then
    \[\lim_{x\to a}\frac{f(a)}{g(a)}=\frac{f'(a)}{g'(a)}\]
    
\end{tcolorbox}

\begin{center}
    \textbf{\large DO NOT USE QUOTIENT RULE!}
\end{center}
\noindent\textbf{Examples:}
\begin{questions}
    \question $\displaystyle \lim_{x\to0}\frac{3x-\sin x}{x}$
    \vspace{\stretch{1}}
    
    \question $\displaystyle \lim_{x\to0}\frac{\sqrt{x+1}-1}{x}$
    \vspace{\stretch{1}}
    
    \question $\displaystyle \lim_{x\to0}\frac{\sqrt{x+1}-1-x/2}{x-2}$
    \vspace{\stretch{1}}
    
    \question $\displaystyle \lim_{x\to\pi/2}\frac{1+\sec x}{\tan x}$
    \vspace{\stretch{1}}
\end{questions}



%%make linearization and newton's method take-home packets

\newpage
