\addcontentsline{toc}{subsection}{Volumes of Revolution}
\subsection*{Volumes of Revolution}
\textbf{Problem 1:} Sketch the region bounded by the $x$-axis, the $y$-axis, the line $x=4$, and the line $y=3$. What is the volume of the solid formed when the region is revolved around the $x$-axis?
\vspace{\stretch{1}}

\textbf{Problem 2:} What is the volume of the solid formed by revolving around the $x$-axis the region bounded by the axes and the line $y=-x+4$?
\vspace{\stretch{1}}

\textbf{Problem 3:} Consider the shape of the region that is generated if $y=x^2$ on the interval $[0,\,3]$ is revolved around the $x$-axis. What is the volume of this solid?
\vspace{\stretch{.5}}

\newpage


\begin{tcolorbox}[title= VOLUME OF A SOLID OF REVOLUTION,colframe=black,sharp corners,colback=white,colbacktitle=white,coltitle=black,boxrule=1pt]

     The volume of a solid of known integrable cross section can be found by rotating the region bounded by $y=f(x),\,y=0,\,x=a,$ and $x=b$ about the $x$-axis using the integral
     \[V=\pi\int_a^b\left(f(x)\right)^2\,dx.\]
     This is commonly referred to as the \textbf{disc method}.
    
\end{tcolorbox}
\textbf{Examples:}
\begin{questions}
    \question Find the volume of the solid formed by revolving the region bounded by the graph of $f(x)=\sqrt{\sin x}$ and the $x$-axis $\left(0\le x\le\pi\right)$ about the $x$-axis.
    \vspace{\stretch{1}}
    
    \question Find the volume of the solid formed by revolving the region bounded by $f(x)=2-x^2$ and the $x$-axis about the $x$-axis.
    \vspace{\stretch{1}}
    
    \question Find the volume of the solid formed by revolving the region bounded by $y=x^{2/3},\, x=0$ and $y=1$ about the $y$-axis.
    \vspace{\stretch{1}}
    
    \newpage
    
    \question Find the volume of the solid formed by revolving the region bounded by $f(x)=2-x^2$ and $g(x)=1$ about the line $y=1$.
    \vspace{\stretch{1}}

\end{questions}

\begin{tcolorbox}[title= VOLUME OF A REGION USING WASHERS,colframe=black,sharp corners,colback=white,colbacktitle=white,coltitle=black,boxrule=1pt]

     \[\text{Volume of a region}=\pi\int_a^b\left(\left[R(x)\right]^2-\left[r(x)\right]^2\right)\,dx.\]
     Where $R(x)$ is the radius of the outer washer, $r(x)$ is the inner radius, and the integral represents the height of the \textit{cylinder}.
    
\end{tcolorbox}

\textbf{Examples:}
\begin{questions}
    \question Find the volume of the solid formed by revolving the region bounded by the graphs of $y=\sqrt{x}$ and $y=x^2$ about the $x$-axis.
    \vspace{\stretch{1}}
    
    \question Find the volume of the solid generated by revolving the region bounded by the curves $y=x^2$ and $y=2-x^2$ about the $x$-axis.
    \vspace{\stretch{1}}
    
    \newpage
    
    \question Find the volume of the solid generated by revolving the region bounded the graphs of $y=x^2+1,\,y=0,\,x=0,$ and $x=1$ about the $y$-axis.
    \vspace{\stretch{1}}
    
    \question The region in the first quadrant enclosed by the $y$-axis and the graphs of $y=\cos x$ and $y=\sin x$ is revolved around the $x$ axis to form a solid. Find its volume.
    \vspace{\stretch{1}}
    
    \question The area of the region bounded by the graphs of $y=\sqrt{x}$ and $y=x^2$ in the first quadrant is revolved around the line $x=3$. Find the volume of this solid.
    \vspace{\stretch{1}}
    
\end{questions}

\newpage

\begin{tcolorbox}[title= VOLUME OF KNOWN CROSS SECTION,colframe=black,sharp corners,colback=white,colbacktitle=white,coltitle=black,boxrule=1pt]

    The volume of a solid of known integrable cross section area $A(x)$ from $x=a$ to $x=b$ is the integral of $A$ from $a$ to $b$,
    \[V=\int_a^b A(x)\,dx.\]
    
\end{tcolorbox}
\textbf{Examples:}

\begin{questions}
    \question A pyramid 3m high has congruent triangular sides and a square base that is 3m on each side. Each cross section of the pyramid parallel to the base is a square. Find the volume of the pyramid.
    
    \vspace{\stretch{1}}
    
    \question A mathematician has a paperweight made so that its base is the shape of the region between the $x$-axis and one arch of the curve $y=2\sin x$. Each cross section cut perpendicular to the $x$-axis is a semi-circle whose diameter runs from the $x$-axis to the curve. Find the volume of the paperweight.
    \vspace{\stretch{1}}
\end{questions}



\newpage
