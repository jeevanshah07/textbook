\addcontentsline{toc}{subsection}{The Definite Integral as Area Under the Curve}
\subsection*{The Definite Integral as Area Under the Curve}

\begin{center}
\begin{tikzpicture}[xscale=1.3,yscale=.65,declare function={f(\x)=((1/3)*(\x)^(3)-3*(\x)^(2)+8*\x-3;}]
\coordinate (start) at (.8,{f(.8)});
\coordinate (x0) at (1,{f(1)});
\coordinate (x1) at (2,{f(2)});
\coordinate (x2) at (3,{f(3)});
\coordinate (x3) at (4,{f(4)});
\coordinate (x4) at (5,{f(5)});
\coordinate (end) at (5.05,{f(5.05)});

\draw [-latex] (-0.5,0) -- (6,0) node (xaxis) [below] {$x$};
\draw [-latex] (0,-0.5) -- (0,5) node [left] {$y$};
\foreach \x/\xtext in {1/a , 5/b }
 \draw[xshift=\x cm] (0pt,3pt) -- (0pt,0pt) 
node[below=2pt,fill=white,font=\normalsize]
  {$\xtext$};
\draw[domain=.5:5.3,samples=200,variable=\x,<->,thick] plot ({\x},{f(\x)}); \end{tikzpicture}
\end{center}

The area under the curve can be estimated using $\displaystyle S_n=\sum_{k=1}^n f\left(c_k\right)\cdot\Delta x_k$, a Riemann sum for $f$ on the interval $[a,\,b].$

\begin{tcolorbox}[title= LIMIT DEFINITION OF THE DEFINITE INTEGRAL,colframe=black,sharp corners,colback=white,colbacktitle=white,coltitle=black,boxrule=1pt]

     Let $f$ be continuous on $[a,\,b]$, and let $[a,\,b]$ be partitioned into $n$ subintervals of equal length $\displaystyle \Delta x=\frac{b-a}{n}$. Then the definite integral of $f$ over $[a,\,b]$ is given by
     \[\lim_{n\to\infty}\sum_{k=1}^n f\left(c_k\right)\Delta x,\]
     where each $c_k$ is chosen arbitrarily in the $k^{\text{th}}$ subinterval.
    
\end{tcolorbox}

\noindent\textbf{Think!} Interpret the above definition and translate into less "mathy" talk:
\vspace{\stretch{.6}}

%actually build out this section and add more examples/explanations/etc.

\textbf{Using the Notation}\\
The interval $[-1,\,3]$ is partitioned into $n$ subintervals of equal length $\Delta x=4/n$. Let $m_k$ denote the midpoint of the $k^{\text{th}}$ subinterval. Express the limit $\displaystyle\lim_{n\to\infty}\sum_{k=1}^n \left( 3\left(m_l\right)^2-2m_k +5 \right)\Delta x$ as an integral.
\vspace{\stretch{.5}}

\newpage

