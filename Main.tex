% Dedicated to Shane Carey for showing me the beauty of mathmatics
% Start: 06/03/2024, 6:04PM
% End:

\documentclass[addpoints]{exam}
\usepackage{amsmath}
\usepackage{amsfonts}
\usepackage{tcolorbox}
\usepackage{tikz}
\usepackage{pgfplots}
\usepackage{mdframed}
\usepackage{tocloft}
\usepackage{tocbibind}
\usepackage{hyperref}
\usepackage{array}
\usepackage{gensymb}


\renewcommand{\arraystretch}{1.2}

\newcommand{\st}{\text{ s.t. }}
\newcommand{\qspace}{\vspace{\stretch{1}}}

\hypersetup{
    colorlinks=true,
    linktoc=subsection,
    linkcolor=black,}

%=============================================
        %POINTS FORMATTING
%=============================================
\marksnotpoints 
%if you would prefer the exam to say "points" instead of "marks", you can delete the above line.

%\pointformat{\textbf{(\thepoints)}} 
%delete comment if you want bold points

\pointsinrightmargin
%\marginpointname{ \points}
\bracketedpoints
%delete comment if you want point values to be printed in the right margin instead of at the start of the question. The same command with the "right" places the point values on the left margin.

%\


%=============================================
        %HEADER/FOOTER FORMATTING
%=============================================
\pagestyle{headandfoot}
%\firstpageheadrule
\runningheadrule
\firstpageheader{}{}{}
\runningheader{Calculus}{}{Mr. Shah}
\firstpagefooter{}{}{}
\runningfooter{ }{\thepage}{ }

%=============================================
        %THE COVER PAGE
%=============================================
\begin{document}

\begin{titlepage}
    \begin{tcolorbox}[valign=center]
        {\begin{center}
        \hspace{0pt}
        \vfill
        \fontsize{45}{45}\scshape Calculus\\ 
        \fontsize{30}{30}\scshape and other higher level mathamatics
        \vfill
        \hspace{0pt}
        \end{center}}
    \end{tcolorbox}
\end{titlepage}

\newpage

\begin{center}
  \thispagestyle{empty}
  \vspace*{\fill}
  Dedicated to Shane Carey, who showed me the beauty of mathematics
  \vspace*{\fill}
\end{center}

\newpage

\tableofcontents 

\newpage 

\phantomsection\addcontentsline{toc}{section}{Introduction: Thinking Mathematically}
\phantomsection\addcontentsline{toc}{subsection}{Logical Calculus}
\section*{Introduction}
\subsection*{Thinking Mathematically}

Despite this being a calculus textbook I will actually start off by teaching something normally taught in a \textit{Discrete Mathematics} course. The first few sections of a discrete course usually go over mathematical logic and proof writing, and here I intended to give you a brief overview (a sparknotes version, if you will) of that. Why you may ask? Simply put, I think that logic (the mathematical sort in specific) is necessary, if not vital, for success not just in math, but also in life.

\phantomsection\addcontentsline{toc}{subsection}{Introduction to Logic}
\subsection*{Introduction to Logic}
Before we begin with the basics, there first something even more basic we must cover. Oftentimesin logic we will create statements full of symbols and it's important to note that the end goal is usually to evaluate if the statement is true or false given a certain set of inputs. In order to abstractly represent this we will use \textit{statement variables}. Statement variables are simply placeholder variables in a statement that can represent either a value of \textbf{true} or \textbf{false}. Now, let's begin with the basics:
\begin{tcolorbox}[title= LOGICAL AND,colframe=black,sharp corners,colback=white,colbacktitle=white,coltitle=black]
  \large\textbf{Logical AND ($\land$)} \\
  \normalsize Logical AND works exactly how you might expect it to: given two inputs, $p$ and $q$, both $p$ AND $q$ must be true for the output to also be true. Logical AND is symbolized using the wedge: $\land$\footnote{In other texts, AND may also be symbolized through multiplicatio: $p*q\equiv\,pq\equiv\,p\land\,q$}. Thus, we can write $p \land q$ which is read as "$p$ and $q$". The truth table\footnote{A truth table is a way to represent all possible truth values for a given statement} for AND look like the following: 
  \vspace{0.01in}
  \begin{center}
    \begin{tabular}{ |c|c|c| } 
      \hline
      $p$ & $q$ & $p \land q$ \\ 
      \hline
      T & T & T \\ 
      T & F & F \\ 
      F & T & F \\ 
      F & F & F \\ 
      \hline
    \end{tabular}
  \end{center}
  \vspace{0.01in}
\end{tcolorbox}

\begin{tcolorbox}[title= LOGICAL OR,colframe=black,sharp corners,colback=white,colbacktitle=white,coltitle=black]
  \large\textbf{Logical OR ($\lor$)} \\
  \normalsize Logical OR works, again, how you would probably expect it, given a statement $p \lor q$ (read "p or q"), \textit{either} p OR q must be true for the output to be true. Logical OR is symbolized using the upside-down wedge: $\lor$\footnote{Similar to AND, OR might be represented through addition: $p+q\equiv\,p\lor\,q$}. Thus, we can write $p \lor q$ which is read "p or q". Using the truth table for AND and the above information about logical or, fill out the below truth table for OR:
  \vspace{0.01in}
  \begin{center}
    \begin{tabular}{ |c|c|c| } 
      \hline
      $p$ & $q$ & $p \lor q$ \\ 
      \hline
      T & T &   \\ 
      T & F &   \\ 
      F & T &   \\ 
      F & F &   \\ 
      \hline
    \end{tabular}
  \end{center}
  \vspace{0.01in}
\end{tcolorbox}

\newpage 


\begin{tcolorbox}[title= LOGICAL NOT,colframe=black,sharp corners,colback=white,colbacktitle=white,coltitle=black]
\large \textbf{LOGICAL NOT} \\ 
  \normalsize Logical not, is very easy to understand. Simply put, the not operator just negates the current value of a variable. If the current value is true then the negated value is false, and vice versa. Logical NOT can by symbolized using $\sim$ or $\lnot$\footnote{NOT, may also be symbolized through an exclimation point: $!p\equiv\,\lnot\,p$}. Thus, we can write $\lnot\,p$ which is read "not p". Once again, the truth table for NOT is left as an exercise to the reader:
  \vspace{0.01in}
  \begin{center}
    \begin{tabular}{ |c|c| } 
      \hline
      $p$ & $\lnot p$ \\ 
      \hline
      T &  \\ 
      T &  \\ 
      F &  \\ 
      F &  \\ 
      \hline
    \end{tabular}
  \end{center}
  \vspace{0.01in}
\end{tcolorbox}

Now, before we work some examples, let's quick take note of the logical order of operations:

\begin{tcolorbox}[title= LOGICAL ORDER OF OPERATIONS,colframe=black,sharp corners,colback=white,colbacktitle=white,coltitle=black]
  \begin{enumerate}
    \item NOT gets evaluated first 
    \item AND second 
    \item OR is the last evaluated
  \end{enumerate}
  Just like in normal algebra, parenthesis can be used to override the order of operations. For example, in the statement: $(p \lor q) \land r$, the parenthesis are used to show that $p \lor q$ should be evaluated first.
\end{tcolorbox}

\textbf{Examples}: Use a truth table to evaluate the truth values of each statement
\begin{questions}
  \begin{minipage}{0.45\linewidth}
    \question $\lnot(p \lor q)$ \\ 
    \vspace{.01in}
      \begin{tabular}{ |c|c|c|c| } 
        \hline
        $p$ & $q$ & $p \lor q$ & $\lnot(p \lor q)$ \\ 
        \hline
        T & T & & \\ 
        T & F & & \\ 
        F & T & & \\ 
        F & F & & \\ 
        \hline
    \end{tabular}
    \vspace{0.01in}
  \end{minipage}
  \hfill
  \begin{minipage}{0.45\linewidth}
    \question $p \land \lnot q$ \\
      \vspace{.01in}
      \begin{tabular}{ |c|c|c|c| }
        \hline 
        $p$ & $q$ & $\lnot q$ & $p \land \lnot q$ \\
        \hline 
        T & & & \\
        T & & & \\
        F & & & \\
        F & & & \\
        \hline
      \end{tabular}
  \end{minipage}  

  \vspace{\stretch{1}}
    
  \question $(p \land q) \land r$ \\
  \vspace{0.1in}
  \begin{tabular}{ |c|c|c|c|c| }
    \hline 
    $p$ & $q$ & $r$ & $p \land q$ & $(p \land q) \land r$ \\
    \hline 
    T & T &T& & \\
    T & T &F& & \\
    T & F &T& & \\
    T & F &F& & \\
    F & T &T& & \\
    F & T &F& & \\
    F & F &T& & \\
    F & F &F& & \\
    \hline
  \end{tabular}
\end{questions}

\newpage 
\phantomsection\addcontentsline{toc}{subsection}{Quantifiers}
\subsection*{Quantifers}
As well as conditional operators, we have quantifiers which we can use to represent general statements about a certain set of objects. It's best to get right into it: 

\begin{tcolorbox}[title=UNIVERSAL QUANTIFIER,colframe=black,sharp corners,colback=white,colbacktitle=white,coltitle=black]
  The universal quantifer, $\forall$, is used to represent a shared truth value amongst \textit{all} values in a given domain. For example, we could say $\forall x \in \mathbb{R}, x * 0 = 0$\footnote{The $\in$ symbol means 'contained in'}\footnote{$\mathbb{R}$ is the set of all real numbers}, this statement would read "\textit{for all} real numbers $x$, $x * 0 = 0$". The formal defintion of the universal quantifer looks something similar to the following: 
  \begin{center}
    Given a statement $Q(x)$ and the domain of $x$ to be $D$, the \textbf{universal statement} $\forall x \in D, Q(x)$\footnote{Note that the $Q(x)$ on its own after the comma is implied to mean that $Q(x)$ is true} is said to be true if, and only if, $Q(x)$ is true for \textit{every} $x$ in $D$. The statement is said to be false if $Q(x)$ is false for \textit{at least one} $x$ in D.
  \end{center}
\end{tcolorbox}

\begin{tcolorbox}[title=EXISTENTIAL QUANTIFIER,colframe=black,sharp corners,colback=white,colbacktitle=white,coltitle=black]
  The existential quantifer, $\exists$, is used to represent a truth value for \textit{at least \textbf{one}} value in a given domain. For example, we could say $\exists x \in \mathbb {R} \text{ such that } \mathrm{\textbf{e}}^{x} = 1$\footnote{The abbriviation 's.t.' is often used in place of 'such that' and will be used going forwards}, which reads "\textit{there exists} a real number, $x$ such that $\mathrm{\textbf{e}}^{x} = 1$". A more formal definiton can be found below:
  \begin{center}
    Given a statement $Q(x)$, and the domain of $x$ to be $D$, the \textbf{existential statement} $\exists x \in D \text{ such that } Q(x)$ is said to be true if, and only if, $Q(x)$ is true for \textit{at least} one $x$ contained in $D$. The statement is said to be false if, and only if, $Q(x)$ is false \textit{for every} $x$ in $D$.
  \end{center}
\end{tcolorbox}

\textbf{For each question, rewrite the statement using the universal or existential quantifier} \\ 
Let $\mathbb{R}$ be the set of real numbers, $\mathbb{N}$ be the set of natural numbers, and $\mathbb{Q}$ be the set of rational numbers

\begin{questions}
  \begin{minipage}{0.45\linewidth}
    \question Every real number times 1 equals itself
  \end{minipage}
  \hfill 
  \begin{minipage}{0.45\linewidth}
    \question There exists a natural number that is both even and prime
  \end{minipage} 
  \vspace{\stretch{1}}

  \begin{minipage}{0.45\linewidth}
    \question Every rational number times it's reciprocal equals 1
  \end{minipage}
  \hfill 
  \begin{minipage}{0.45\linewidth}
    \question For all real numbers $x$, there exists another real number, $y$, such that $x+y=0$
  \end{minipage}
  \vspace{\stretch{1}}
\end{questions}

\newpage 

\begin{tcolorbox}[title=COMBINING QUANTIFERS,colframe=black,sharp corners,colback=white,colbacktitle=white,coltitle=black]
  As you saw, the final exercise on the previous page required the use of both the universal and the existential quantifer, which is not an uncommon occurence. When we combine two quantifers in a statement they are intepreted \textbf{in the order they occur}. Thus the statements $\forall x \in \mathbb{Z}, \exists y \in \mathbb{Z} \text{ s.t. } P(x, y) $ and $\exists x \in \mathbb{Z} \text{ s.t. } \forall y \in \mathbb{Z}\text{, } P(x, y) $ have very different meanings. This leads us to the following: \textit{switching the order of different quantifers may (and often will) change the meaning of a statement}. However, if two quantifers are of the same type, then switching the order \textbf{will not} change the value of the statement: $\forall x \in \mathbb{Z}\text{, } \forall y \in \mathbb{Q}, P(x, y) \equiv \forall y \in \mathbb{Q}\text{, } \forall x \in \mathbb{Z}, P(x, y)$ and $\exists x \in \mathbb{Q} \text{ s.t. } \exists y \in \mathbb{A} \text{ s.t. } P(x, y) \equiv \exists y \in \mathbb{A} \text{ s.t. } \exists x \in \mathbb{Q} \st P(x, y)$\footnote{$\mathbb{A}$ is the set of algebraic numbers, for more information see \href{https://en.wikipedia.org/wiki/Algebraic_number}{\underline{this link}} or \href{https://www.mathsisfun.com/numbers/algebraic-numbers.html}{\underline{this link}}}. \\ 
  \small \textit{* Note that in these examples $P(x, y)$ is a predicate, which contains variables and becomes a statement when specific values are subsituted for the variables}
\end{tcolorbox}

Before we begin our calculus journey, there is one final logic topic I would like to cover: implication. 

\begin{tcolorbox}[title=IMPLIES,colframe=black,sharp corners,colback=white,colbacktitle=white,coltitle=black]
  Implications are used for conditional statements and is represented by an arrow: $\to$. For example: if it it snowing, then it is below 32\degree F\footnote{Assume, in this case, that in order for it be snowing it \textit{must} be below 32\degree F}. We can rewrite this symbolically by representing the statement 'it is snowing' with 'S' and the statement 'it is raining' with 'R'. Thus we get: $S \to R$ which would be read as "If $S$ then $R$". The general conditional statement is $H \to C$ or "if hypothesis, then conclusion". The conditional statement is true if, and only if, both the hypothesis and the conclusion are true, or if the hypothesis is false. The second part may through some for a loop, but consider the earlier example. If it is \textit{not} snowing, then it must also not be below 32\degree F, which is another true statement. Or consider a different perspective: if it \textit{is} snowing, but it \textit{is not} below 32\degree F, then we have a contradiction, and so our statement must be false. With all this in mind, see the truth table for the conditional statement: 
  \vspace{0.01in}
  \begin{center}
    \begin{tabular}{ |c|c|c| } 
      \hline
      $p$ & $q$ & $p \to q$ \\ 
      \hline
      T & T & T \\ 
      T & F & F \\ 
      F & T & T \\ 
      F & F & T \\ 
      \hline
    \end{tabular}
  \end{center}
  \vspace{0.01in}
\end{tcolorbox}

For the following examples, rewrite the statement without using any symbols: \\
\small \textit{Recall that $\mathbb{R}$ is the set of all real numbers, $\mathbb{Q}$ is the set of all rational numbers, and $\mathbb{N}$ is the set of all natural numbers}
\begin{questions}
  \question $\forall x \in \mathbb{R}\text{, } \exists y \in \mathbb{R} \st x * y = x$
  \qspace 
  
  \question $\exists a \in \mathbb{R} \st \forall b \in \mathbb{R}\text{, } a * b = a$ 
  \qspace

  \question $p \in \mathbb{N} \to p \in \mathbb{Q}$
  \qspace
\end{questions}
\end{document}
