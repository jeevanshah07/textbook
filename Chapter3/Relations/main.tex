\subsection{Relations}
\begin{definition}{Relations}{}
    Let $X$ and $Y$ be sets. A relation $R$ from $X$ to $Y$ is a subset of $X \times Y$. An ordered pair $(x, y)$ is contained in $R$ if, and only if, $x$ is related to $y$, denoted $x~R~ y$ or $x \sim_R y$ where $x \in X$ and $y \in Y$. If $X = Y$ then it is simply a relation on $X$.
\end{definition}

\begin{example}{Example relations}{}
    \begin{itemize}
        \item Define a relation $E$ from $\ZZ$ to $\RR$ such that $x \sim_E y \Leftrightarrow x - y > 0$. 
        \begin{itemize}
            \item $5 \sim_E 3$ since $5-3 = 2 > 0$
            \item $5 \sim_E \pi$ since $5 - \pi > 0$
            \item $\pi \not\sim_E 5$ since $\pi - 5 < 0$
        \end{itemize}

        \item Define a relation $T$ from $A = \braces{0, 1, 2, 3}$ to $B = \braces{4, 5, 6, 7}$ by the rule $x \sim_T y \Leftrightarrow x \mid y$ 
        \begin{itemize}
            \item $1 \sim_T y \hspace{0.2cm} \forall y \in B$ since $y = 1k$ has solutions, namely $k=y$
            \item $0 \not\sim_T y \hspace{0.2cm} \forall y \in B$ since $y = 0k$ has no solutions besides $0$ and $0 \notin B$
        \end{itemize}
    \end{itemize}
\end{example}

\begin{question}{Finding relations}{}
    Find every element in $B$ that is related to $2 \in A$ for the relation defined in the above example.
\end{question}

\subsection{Equivalence Relations}
\begin{definition}{Equivalence Relations}{}
    Let $R$ be a relation on $X$. $R$ is an equivalence relation if, and only if, it is \textit{symmetric}, \textit{transative}, and \textit{reflexive}.
    \begin{itemize}
        \item A relation is \textit{symmetric} if $a \sim_R b \Rightarrow b \sim_R a$ for $a, b \in X$.
        \item A relation is \textit{transative} if $a \sim_R b$ and $b \sim_R c \Rightarrow a \sim_R c$ for $a, b, c \in X$
        \item A relation is \textit{reflexive} if $a \sim_R a$ for $a \in X$.
    \end{itemize}
    To prove that a relation is an equivalence relation you must show that it is reflexive, symmetric, and transitive.
\end{definition}

\begin{question}{Relation between Circuits}{circ}
    Let $C$ be the set of all digital circuits that can be created using basic logic gates such as AND, OR, NOT, NOR, NAND, and XOR. Define a relation $\sim_E$  such that for any two circuits $a, b \in C$, $a \sim_E b$ if, and only if, $a$ and $b$ have the same truth/output tables. Prove that $\sim_E$ is an equivalence relation.
\end{question}

\begin{tcolorbox}[colback=pink]
\begin{theorem}
    $\sim_E$, as defined in~(\ref{th:circ})~is an Equivalence Relation
\end{theorem}
\begin{proof}
    Let $C$ and $\sim_E$ be defined as in~(\ref{th:circ}). \\
    \hspace*{2em}\underline{Proof of Reflexivity:} Let $a \in C$. Then $a \sim_E a$ since $a$ must have the same truth table as itself. \\
    \hspace*{2em}\underline{Proof of Symmetry:} Let $a, b \in C$. Then $a \sim_E b$ and $b \sim_E a$ since if $a$ and $b$ have the same truth tables then $b$ and $a$ must have the same truth tables. \\
    \hspace*{2em}\underline{Proof of Transativity:} Let $a, b, c \in C$ such that $a \sim_E b$ and $b \sim_E c$. Then, $a \sim_E c$ since if $a$ and $b$ have the same truth tables and $b$ and $c$ have the same truth tables, $a$ and $c$ must also have the same truth tables.
    Therefore, since $\sim_E$ is reflexive, symmetric, and transitive, $\sim_E$ is an equivalence relation on $C$.
\end{proof}
\end{tcolorbox}

\subsubsection{Equivalence Classes}
There may be times when we want to consider the set of all items that are related to a single item. Luckily for us, we have something that does exactly this.

\begin{definition}{Equivalence Classes}{}
    Let $\sim_R$ be a relation on a set $X$. Define the an equivalence class as the following set:
    \begin{equation*}
        [a] = \braces{ x \in X \mid x \sim_R a}
    \end{equation*}
    This is the set of all elements in $X$ that are related to $a$.
\end{definition}

\begin{tcolorbox}[colback=yellow!60]
\begin{lemma}\label{lma:equiv}
    If $b \sim_R a$ then $[a] = [b]$
\end{lemma}
\begin{proof}
    To show that any two sets, $A$ and $B$, are equivalent we must show that $A \subseteq B$ and $B \subseteq A$
    \hspace*{2em} \underline{Proof that $[a] \subseteq [b]$:} Let $x \in [a]$. Then, by definition of equivalence classes $x \sim_R a$. But $a \sim_R b$ so $x \sim_R b$ since $\sim_R$ is transitive. Thus, $x \in [b]$. \\
    \hspace*{2em} \underline{Proof that $[b] \subseteq [a]$:} Trivial and left as an exercise to the reader. \\
    Thus, since $[a] \subseteq [b]$ and $[b] \subseteq [a]$, $[a] = [b]$.
\end{proof}
\end{tcolorbox}

Now following~(\ref{lma:equiv})~it shouldn't be much of a surprise that when we wish to discuss the equivalence classes of a relation we often don't want to talk about \textit{all} of them. Instead we talk about the \textbf{distinct} equivalence classes of a relation. 
\begin{definition}{Distinct Equivalence Classes}{}
    The \textbf{distinct equivalence classes} of a relation are the first $n$ equivalence classes such that no two equivalence classes are equivalent. Standardly we start with $0$ and work our way up to $n$.
\end{definition}

\begin{example}{Equivalence Classes of $\mathop{\bmod}{2}$}{}
    Let $\sim_R$ be a relation on $\ZZ$ such that $m \sim_R n \Leftrightarrow 2 \mid (m - n)$. We can find the distinct equivalence classes of $\sim_R$ by noticing that $2 \mid (m-n) \Leftrightarrow m \equiv n \pmod{2}$. With this in mind,
    \begin{align*}
        [a] &= \braces{x \in \ZZ \mid x \sim_R a} \\
            &= \braces{x \in \ZZ \mid 2 \mid (x - a)} \\
            &= \braces{x \in \ZZ \mid x \equiv a \pmod{2}}
    \end{align*}
    Hopefully, from here it should be pretty clear what our two distinct equivalence classes are:
    \begin{align*}
        [0] &= \braces{\dots, -4 ,-2, 0, 2, 4, \dots} \\
        [1] &= \braces{\dots, -3, -1, 1, 3, 5 \dots} 
    \end{align*}
\end{example}

\begin{question}{Equivalence classes of congruence modulo $n$}{}
    What are the equivalence classes for the relation $\sim_T$ on $\ZZ$ where $m \sim_T n \Leftrightarrow 3 \mid (m-n)$? What about $m \sim_G n \Leftrightarrow 4 \mid (m-n)$? Can you generalize for $x \sim_C y \Leftrightarrow n \mid (x-y)$?
\end{question}

\begin{tcolorbox}[colback=pink]
\begin{theorem}[Equivalence Classes of Congruence modulo $n$ relation]
    Let $\sim_R$ be a relation on $\ZZ$ such that $a \sim_R b \Leftrightarrow n \mid (a-b) \Leftrightarrow a \equiv b \pmod{n}$. The distinct equivalence classes of $\sim_R$ will be 
    \begin{align*}
        [0] &= \braces{x \in \ZZ \mid x \equiv 0 \pmod{n}} = \braces{x \in \ZZ \mid x = nk, \hspace{0.2cm} k \in \ZZ} \\
        [1] &= \braces{x \in \ZZ \mid x \equiv 1 \pmod{n}} = \braces{x \in \ZZ \mid x = nk + 1, \hspace{0.2cm} k \in \ZZ} \\
        &\vdots \\
        [n-1] &= \braces{x \in \ZZ \mid x \equiv n-1 \pmod{n}} = \braces{x \in \ZZ \mid x = nk + (n-1), \hspace{0.2cm} k \in \ZZ}
    \end{align*}
\end{theorem}
\begin{proof}
    Left as an exercise to the reader. {\tiny\textit{(Hint: Use the quotient remainder theorem)}}
\end{proof}
\end{tcolorbox}